\documentclass[english]{article}
\usepackage{amsmath,amssymb}
\usepackage{fullpage}
\usepackage{enumitem}
\usepackage{multicol}
\usepackage{float}
\usepackage{kotex}
\usepackage{listings}
\usepackage[dvipsnames]{xcolor}
\usepackage{graphicx}


\lstdefinelanguage{Kotlin}{
  comment=[l]{//},
  commentstyle={\color{gray}\ttfamily},
  emph={filter, first, firstOrNull, forEach, lazy, map, mapNotNull, println},
  emphstyle={\color{OrangeRed}},
  identifierstyle=\color{black},
  keywords={!in, !is, abstract, actual, annotation, as, as?, break, by, catch, class, companion, const, constructor, continue, crossinline, data, delegate, do, dynamic, else, enum, expect, external, false, field, file, final, finally, for, fun, get, if, import, in, infix, init, inline, inner, interface, internal, is, lateinit, noinline, null, object, open, operator, out, override, package, param, private, property, protected, public, receiveris, reified, return, return@, sealed, set, setparam, super, suspend, tailrec, this, throw, true, try, typealias, typeof, val, var, vararg, when, where, while},
  keywordstyle={\color{NavyBlue}\bfseries},
  morecomment=[s]{/*}{*/},
  morestring=[b]",
  morestring=[s]{"""*}{*"""},
  ndkeywords={@Deprecated, @JvmField, @JvmName, @JvmOverloads, @JvmStatic, @JvmSynthetic, Array, Byte, Double, Float, Int, Integer, Iterable, Long, Runnable, Short, String, Any, Unit, Nothing},
  ndkeywordstyle={\color{BurntOrange}\bfseries},
  sensitive=true,
  stringstyle={\color{ForestGreen}\ttfamily},
}

\lstset{language=kotlin} 

\newcommand{\handout}[4]{
   \renewcommand{\thepage}{}
   \noindent
   \begin{center}
   \framebox{
      \vbox{
    \hbox to 6in { {\bf CSE 4190.101: Discrete Mathematics}
         \hfill #2 }
       \vspace{4mm}
       \hbox to 6in { {\Large \hfill #1  \hfill} }
       \vspace{2mm}
       \hbox to 6in { {\emph{#3} \hfill #4} }
      }
   }
   \end{center}
   \vspace*{4mm}
}

\newcommand{\homework}[4]{\handout{Problem Set #1}{#2}{Instructor: #3}{{\bf Due on:} #4}}

\begin{document}
\homework{1}{Fall 2021}{Yongsoo Song}{Oct 04, 2021}
Student No: 2019-13674

Please submit your answer through eTL. It should be a {{single PDF}} file (not jpg), either typed or scanned.
{\bf Please include your student ID and name (e.g. 2020-12345 YongsooSong.pdf).}
You may discuss with other students the general approach to solve the problems, but the answers should be written in your own words. You should cite any reference that you used, and mention what you used it for. You should follow the academic integrity rules that are described in the course information.

\vspace{.5cm}
\hrule
\vspace{.5cm}

\section*{Problem 1 (5 points)}

Let the following statements be given:
\begin{itemize}
\item [{(p)}] {}``You can install the program''
\item [{(q)}] {}``Your computer has \textbf{less} than 2GB of RAM''
\item [{(r)}] {}``Your computer has 20GB of free disk space''
\end{itemize}  
\begin{enumerate}
\item Translate the following statement into symbols of formal logic:
  \begin{quote}
  ``In order to install the program you need at least 2GB of RAM and 20GB of free disk space''
  \end{quote}
  $ p \to \lnot q \land r $
\item Give the \emph{converse} of this statement in the symbols of formal logic. 

  $ \lnot q \land r \to p $
\item Give the \emph{converse} in English

  If you have at least 2GB of RAM and 20GB of free disk space you can install the program.
\item Give the \emph{contrapositive} (of the original statement) in the symbols of formal logic

$ q \lor \lnot r \to \lnot p$
\item Give the \emph{contrapositive} in English

  If your computer has less than 2GB of RAM or Your computer does not have 20GB of free disk space then you cannot install the program.
\end{enumerate}

\section*{Problem 2 (5 points)}
Determine if the same is true for the implication operation ($\to$)
i.e., is the expression $(P\to Q)\to R$  logically equivalent to $P \to (Q \to R)$?
Prove or disprove using the truth table method. 

\begin{table}[H]
  \begin{center}
  \begin{tabular}{c c c | c | c | c | c | c | c}
      P & Q & R & $(P\to Q)$ & R & $(P\to Q)\to R$ & $(Q\to R)$ & $P\to (Q\to R)$ & same \\
      \hline
      0 & 0 & 0 & 1 & 0 & 0 & 1 & 1 & False \\ 
      0 & 0 & 1 & 1 & 1 & 1 & 1 & 1 & True \\ 
      0 & 1 & 0 & 1 & 0 & 0 & 0 & 1 & False \\ 
      0 & 1 & 1 & 1 & 1 & 1 & 1 & 1 & True \\ 
      1 & 0 & 0 & 0 & 0 & 1 & 1 & 1 & True \\ 
      1 & 0 & 1 & 0 & 1 & 1 & 1 & 1 & True \\ 
      1 & 1 & 0 & 1 & 0 & 0 & 0 & 0 & True \\ 
      1 & 1 & 1 & 1 & 1 & 1 & 1 & 1 & True \\ 
      \hline
  \end{tabular}
  \end{center}
  \caption{\label{tab:table-name} truth table}
\end{table}

They are not equivalent.

\section*{Problem 3 (5 points)}
Construct a compound proposition that asserts that every cell of a $9 \times 9$ Sudoku puzzle contains at least one number (using the propositions $p(i,j,n)$ defined in the lecture note).

$$\bigwedge_{i=1}^9\bigwedge_{j=1}^9\bigvee_{n=1}^9 p(i,j,n)$$

\section*{Problem 4 (10 points)}

You are on a treasure island, and find a note with the following hints:
\begin{enumerate}
\item If this house is next to a lake, then the treasure is not in the kitchen
\item If the tree in the front yard is an elm, then the treasure is in the kitchen
\item This house is next to a lake
\item Either the tree in the front yard is an elm or the treasure is not buried under the flagpole
  (or both. This ``or'' is inclusive.)
\item The treasure is either under the flagpole or in the garage.
  (but not both. This is an exclusive or, there is only one treasure!)
\end{enumerate}

In order to find where the treasure is, do the following:

\paragraph{(a)}
Identify 5 atomic statements that can be used to express the above statements in logical form. (Atomic statements should \textbf{not} contain any
word corresponding to a logical connective.)
List the 5 statements in the order in which they 
appear in the above messages, and call them $L,K,E,P,G$. Express each of the above compound statements (1-5) using $L,K,E,P,G$ and 
logical connectives.

\begin{enumerate}
  \item [{{L}}] This house is next to a lake.
  \item [{{K}}] The treasure is in the kitchen.
  \item [{{E}}] The tree in the front yard is an elm.
  \item [{{P}}] The treasure is buried under the flagpole.
  \item [{{G}}] The treasure is in the garage.
\end{enumerate}

\begin{enumerate}
  \item $L \to \lnot K$
  \item $E \to K$
  \item $L$
  \item $E \lor \lnot P$
  \item $P \oplus G $
\end{enumerate}

\paragraph{(b)} Determine where the treasure is, and justify your answer.
\begin{align}
  & L & \text{by 3}\\
  & L \to \lnot K & \text{by 1}\\
  & \lnot K \to \lnot E & \text{by contrapositive of 2}\\
  & \lnot E \to \lnot P & \text{by 4}\\
  & \lnot P \to G & \text{by 5}
\end{align}
$L \to \lnot K \to \lnot E \to \lnot P \to G$

Garage.

\section*{Problem 5 (10 points)}

Let $D = \{-48,-14,-8,0,1,3,16,23,26,32,36\}$.
Determine which of the following statements are true and which are false.
Provide a counterexample for the statements that are false.
In all statements, the variables $x,y$ range over the set $D$.

\begin{enumerate}
\item $\forall x$, if $x$ is odd then $x>0$.
  True.
\item $\forall x$, if $x<0$ then $x$ is even
  True.
\item $\forall x$, $\exists y$ such that $y > x$
  False; x = 36
\item $\forall x$, ($x$ is even or $\exists y, y>x$)
  True
\item $\forall x$, ($x$ is odd  or $\exists y, y>x$)
  False; x = 36
\end{enumerate}

\section*{Problem 6 (5 points)}

Prove the following statement by contraposition:

\begin{quote}
  ``For any integer $n$, if $(n^2+n+1)$ is even, then $n$ is odd.''
\end{quote}

contraposition:
\begin{quote}
  ``For any integer $n$, if $n$ is even, then $(n^2+n+1)$ is odd.''
\end{quote}

\begin{align*}
  & \text{let } n = 2k, k \in \mathbb{Z} \\
  & n^2 + n + 1  \\
  &= (2k)^2 + 2k + 1 \\
  &= 2(2k^2 + k) + 1 \\
  & \text{is odd.}
\end{align*}



\section*{Problem 7 (10 points)} Suppose that $a$, $b$ and $c$  are odd integers. Assume that a real number $x$ satisfies the equation $ax^2+bx+c=0$. Prove by contradiction that $x$ is irrational.

Suppose that the root $x$ is rational. $x = \frac pq, p, q \in \mathbb{Z}, GCD(p, q) = 1$
Then 
\begin{align*}
  ax^2 + bx + c & \\
  = a\left(\frac pq\right)^2 + b\left(\frac pq \right) + c & \\
  = a\frac{p^2}{q^2} + b\frac pq + c &= 0 \\
  \therefore ap^2 + bpq + cq^2 &= 0
\end{align*}
\begin{itemize}
  \item Let p is odd and q is odd. Then $ap^2$ is odd, $bpq$ is odd, and $cq^2$ is odd, $ap^2 + bpq + cq^2$ is odd. and 0 is not odd. contradiction.
  \item Let p is odd and q is even. Then $ap^2$ is odd, $bpq$ is even and $cq^2$ is even, $ap^2 + bpq + cq^2$ is odd. and 0 is not odd. contradiction.
  \item Let p is even and q is even. $GCD(p, q) \neq 1$. contradiction.
\end{itemize}
Every 3 cases have contradictions. So $x$ is not rational.

\section*{Problem 8 (10 points)} Prove or disprove that you can tile a $10 \times 10$ checkerboard using straight tetrominoes.
보드를 네 가지 색으로 색칠한다. 2 x 2 영역을 놓았을 때, 시계방향으로 Yellow Red Blue Green 이렇게 네 개의 색으로 색칠이 되어 있다. 이 2 x 2 영역이 가로 5개 세로 5개 총 25개의 패치가 있다. 각 색은 25개씩이다. 귀퉁이는 각각 Yellow Red Green Blue로 채워져 있다.
\begin{figure}[H]
  \includegraphics[width=8cm]{checkboard.png}
\end{figure}
이 상태에서 테트로미노를 놓는다. 어떤 테트로미노는 Yellow 2개 Red 2개, 어떤 테트로미노는 Green 두 개, Blue 두 개, 어떤 테트로미노는 Yellow 2개, Green 두 개, 어떤 테트로미노는 Red 두 개, Blue 두 개를 덮을 수 있다. 이외의 경우는 없다.
이 경우 각 색은 25개씩인데, 각각의 테트로미노는 각 색 당 두 개씩의 영역을 처리할 수 있다. 즉 테트로미노들로 짝수 개의 색들을 처리할 수 있는데 각 색은 25개씩이므로 절대로 체커보드를 체울 수 없다.

\section*{Problem 9 (10 points)}

For each of the following statements, determine if it is true or false.
If true, prove it. Otherwise, disprove it by giving a counterexample.

\paragraph{(a)}
For any sets $A,B,C$, if $A\cup C \subseteq B\cup C$, then $A \subseteq B$.

% \begin{align}
%   & x \in A \lor x \in C \rightarrow x \in B \lor x \in C \\
%   \Leftrightarrow & \lnot (x \in A \lor x \in C) \lor (x \in B \lor x \in C) \\
%   \Leftrightarrow & (x \notin A \land x \notin C) \lor (x \in B \lor x \in C)
% \end{align}
counterexample
$x \in A \land x \notin B \land x \in C$ 

\paragraph{(b)}
For any sets $A,B,C$, if $A\cup C \subseteq B\cup C$ and $A\cap C \subseteq B\cap C$,
then $A \subseteq B$.

\begin{align*}
  A &= A\cup (A \cap C) \\
  &\subseteq A\cup(B\cap C) \\
  &=(A \cup B)\cap (A\cup C) \\
  &\subseteq (A\cup B)\cap (B\cup C) \\
  &= B\cup(A \cap C) \\
  &\subseteq B\cup (B\cap C) \\
  &= B
\end{align*}


\section*{Problem 10 (10 points)}
Let $F$ be the set of all nonempty finite sets of integers, and let $f: F\to\mathbb{Z}$
be the function $f(S) = \sum_{x\in S} x$ mapping each set to the sum of its elements.
So, for example, $f(\{1,-7,9\}) = 3$.

\paragraph{(a)} Determine if $f$ is surjective, and prove your answer.

For any $z \in \mathbb {Z}$, there exists $\{z\} \in F$ s. t. $f(\{z\}) = z.$
So f is surjective.

\paragraph{(b)} Determine if $f$ is injective, and prove your answer.
$f(\{\{1, 4\}) = f( \{2, 3\}\}) = 5$
f is not injective.


\section*{Problem 11 (10 points)}

Let $f: \mathbb{N}^2 \to \mathbb{Z}$ be the function $f(x,y) = x-y$,
where $\mathbb{N} = \{1,2,3,4,\ldots\}$ is the set of positive integers.

\paragraph{(a)}
Show that $f: \mathbb{N}^2 \to \mathbb{Z}$ is not a bijection.
(x, y) 가 (2, 1) , (3, 2)이면 둘 다 함숫값이 1이다. 그래서 injective가 아니다.

\paragraph{(b)}
Give a subset $S \subseteq\mathbb{N}^2$ such that
$f: S \to \mathbb{Z}$ is a bijection, and prove the
correctness of your answer.
$ S = \{(1, 1), (1, 2), (1, 3), \cdots, (2, 1), (3, 1), (4, 1) \cdots \} $
\begin{itemize}
  \item Injective:
   For $a \neq b \in \mathbb{N} $
   \begin{align}
    & f(1, a) = 1 -a \neq f(1, b) = 1 - b \\
    & f(a, 1) = a - 1 \neq f(b, 1) = b - 1
   \end{align}
  \item Surjective:
   For $z \in \mathbb{Z}  s. t. f(a, b) = a -b = z, (a, b) \in \mathbb{S}$, When $z > 0$, $a = z+1$, $b = 1$. When $z = 0$, $a = 1$, $b = 1$. When  $z < 0$, $a = 1$, $b = 1 - z$.

\end{itemize}

\section*{Problem 12 (10 points)}

Show that if $S$ is a set, then there does not exist an onto function $f$ from $S$ to $P\mathcal P(S)$, the power set of $S$. Conclude
that $|S| < |\mathcal P(S)|$. [Hint: Suppose such a function $f$ existed and consider the set $T = \{s \in S | s  \notin f (s) \}$.]

\begin{lstlisting}
  val S: Set<Number>
  fun f (v : Number) : Set<Number>
  val PS : Set<Set<Number>>
  val T: Set<Number>
  val a : Number
\end{lstlisting}

Suppose such a function $f$ existed and consider the set $T = \{s \in S | s  \notin f (s) \}$ ($T = \{s | s \in S \land s \notin f(s)\}$). $T \in P\mathcal P(S)$. However, there is no $a \in S$ such that $f(a) = T$.
If $a \in T$, $a \notin f(a)$. So $f(a) \neq T$, because $a \in T$ but $a \notin f(a)$
If $a \notin T$, $a \in f(a)$. So $f(a) \neq T$, because $a \in f(a)$ but $a \notin T$.
So $T \neq f(a) \forall a \in S)$. So f is not surjective.  

\end{document}